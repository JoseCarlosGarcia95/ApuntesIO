% Configuración
\title{Programación lineal}
\university{Universidad de Cádiz}
\author{Antonio J. \textsc{Arriaza Gómez}\\
Manuel \textsc{Muñoz Márquez}\\
Francisco J. \textsc{Navarro Izquierdo}\\
Inmaculada \textsc{Espejo Miranda} \\
J. Carlos \textsc{García Ortega}}

%\advisortwo{Segundo tutor}
\city{Jerez de la Frontera, Cádiz}


% Language
\usepackage[utf8]{inputenc}
\usepackage[spanish]{babel}

% Fuentes
%\usepackage[math]{kurier}
\usepackage[sfdefault,condensed]{roboto}
\usepackage[T1]{fontenc}

\usepackage{blkarray, bigstrut}
\usepackage{multirow}

% Símbolos matemáticos
\def\IR{\ensuremath{\mathbb R}}
\def\IK{\ensuremath{\mathbb K}}
\def\IC{\ensuremath{\mathbb C}}
\def\IN{\ensuremath{\mathbb N}}
\def\IZ{\ensuremath{\mathbb Z}}
\def\IQ{\ensuremath{\mathbb Q}}
\def\IA{\ensuremath{\mathbb A}}

% Tabla simplex (thanks Javi)
\usepackage{multicol}
\usepackage{forloop}
\usepackage{calculator}

\newcounter{ct}
\def\fin{\multicolumn{1}{c}}
	
\newenvironment{simplex}[4]
{ 
	\def\myslines{} %
	\forloop{ct}{0}{\value{ct} < #1}{\expandafter\def\expandafter\myslines\expandafter{%
			\myslines r
		}
	}
	\ADD{#1}{3}{\sol}
	\begin{array}{rr|\myslines|c}
		& \multicolumn{1}{c}{} &#2&\\
		& & #3\\
		\cline{2-\sol}
		#4
		\cline{2-\sol}
		& & }
{\\\end{array}}

% Gráficos
\graphicspath{{graficos/}}
\ifthenelse{\equal{\PDFMODE}{1}}{
  \usepackage{subcaption}
}{}


% Legacy
\newenvironment{formulacion}[4][1]
{ 
  \def\myflines{} %
  \forloop{ct}{-2}{\value{ct} < #3}{\expandafter\def\expandafter\myflines\expandafter{%
      \myflines
      r
    }%
  }
  \begin{array}{c}
    #2 \quad #4\vspace*{0.3cm}\\
    \text{s.a}\;
    \left\{
    \renewcommand{\arraystretch}{1.2}
    \begin{array}{\myflines}}
      {\end{array}\right.
  \end{array}
}