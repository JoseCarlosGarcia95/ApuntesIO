\chapter{Método Simplex}
En el capítulo anterior se ha hecho una introducción de que es la
programación lineal y los distintos teoremas que relacionan los
problemas de optimización con representaciones geométricas.

En este capítulo se introducirá el método simplex, un algoritmo para
poder resolver problemas de programación lineal.

En resumen, el método simplex nos calcula soluciones óptimas de un
problema, teniendo en cuenta:

\begin{enumerate}
  \item Hay que cambiar de soluciones básicas adyacentes en soluciones
  básicas adyacentes.
  \item Conseguir que la solución básica sea factible.
  \item Calcular que variable debe entrar en la base para mejorar la
  función objetivo.
\end{enumerate}

Teniendo en cuenta esto, a continuación se describirán las distintas
fases del método simplex
\section{Las tres fases del método simplex}

\subsection{Pivotar, "saltar" de solución básica factible en solución básica factible}
Sea $[P]$ un PPL en formato estándar:
$$ min(max) c^t x$$
$$
s.a \left\{
  \begin{array}{c}
    Ax=b \\
    x \geq 0
  \end{array}
\right.
$$
$$ A \in \mathcal{M}_{mxn}, b\geq 0, rango(A)=m$$
Podemos suponer sin perdida de generaldiad que la matriz $A$ se puede
escribir como $$[A][x]=[B_{mxn} ~ N_{mx(n-m)}]
\left[
  \begin{array}{c}
    x_B \\
    x_N
  \end{array}
\right]=[b]$$
La matriz $B$ es invertible, por tanto podemos multiplicar por $B^{-1}$ y tenemos:
$$[Id_{mxn} ~ N_{mx(n-m)}]
\left[
  \begin{array}{c}
    x_B \\
    x_N
  \end{array}
\right]=[B^{-1}][b]$$
\begin{ejemplo}
  $$ max/min ~c^t x$$
  $$s.a \left\{
    \begin{array}{c}
      x_1+5x_4=57 \\
      x_1-4 x_4=12 \\
      x_1+x_4 = 5
    \end{array}
  \right.$$
  \textbf{Tabla simplex}
  $$
  \begin{array}{c|cccc|c}
    & x_1 & x_2 & x_3 & x_4 &  \\ \hline
    x_1 & 1 & 0 & 0 & 5 & 57 \\
    x_2 & 0 & 1 & 0 & -4 & 12 \\
    x_3 & 0 & 0 & 1 & 1 & 5 \\ \hline
    & & & & &
  \end{array}
  $$
  Supongamos que $x_4$ mejora a la solución actual. Cambiamos por ejemplo, $x_4$ por $x_3$
  $$
  \begin{array}{c|cccc|c}
    & x_1 & x_2 & x_3 & x_4 &  \\ \hline
    x_1 & 1 & 0 & 0 & 5 & 57 \\
    x_2 & 0 & 1 & 0 & -4 & 12 \\
    x_3 & 0 & 0 & 1 & 1 & 5 \\ \hline
    & & & & \uparrow &
  \end{array}
  $$
  Hacemos operaciones Gaussianas para poner la columna $x_4$ como básica:
  $$
  \begin{array}{c|cccc|c}
    & x_1 & x_2 & x_3 & x_4 &  \\ \hline
    x_1 & 1 & 0 & -5 & 0 & 32 \\
    x_2 & 0 & 1 & 4 & 0 & 32 \\
    x_4 & 0 & 0 & 1 & 1 & 5 \\ \hline
    & & & &  &
  \end{array}
  $$
  Ahora volvemos a la solución básica inicial
  $$
  \begin{array}{c|cccc|c}
    & x_1 & x_2 & x_3 & x_4 &  \\ \hline
    x_1 & 1 & 0 & 0 & 5 & 57 \\
    x_2 & 0 & 1 & 0 & -4 & 12 \\
    x_3 & 0 & 0 & 1 & 1 & 5 \\ \hline
    & & & & &
  \end{array}
  $$
  Ahora queremos que entre en la base la variable $x_4$ y que salga la variable $x_2$
  $$
  \begin{array}{c|cccc|c}
    & x_1 & x_2 & x_3 & x_4 &  \\ \hline
    x_1 & 1 & \frac{5}{4} & 0 & 0 & 72 \\
    x_4 & 0 & -\frac{1}{4} & 0 & 1 & -3 \\
    x_3 & 0 & \frac{1}{4} & 1 & 0 & 8 \\ \hline
    & & & & &
  \end{array}
  $$
\end{ejemplo}
\subsection{Dada una variable no básica que mejore la solución actual, ¿Qué variable no básica debe salir?}
Sea $[P]$ un PPL en formato estándar:
$$ min ~(max) c^tx$$
$$s.a \left\{
  \begin{array}{c}
    Ax=b \\x \geq 0
  \end{array}
\right. $$
La matriz
$$ [A][x]=[b]$$
Podemos reescribirla de forma que
$$[B ~ N]
\left[
  \begin{array}{c}
    x_B \\
    x_N
  \end{array}
\right]=[b]
$$
Podemos suponer sin pérdida de generalidad que
la $$B=I_{mxn}=\left(
  \begin{array}{cccc}
    1 & ... & ... & ... \\
    0 & 1 & ... & ... \\
    ... & ... & ... & ... \\
    ... & ... &  0  & 1
  \end{array}
\right)
$$
Podemos escribir en formato tabla en formato simplex:

$$\left[
  \begin{array}{cccc}
    1 & ... & ... & ... \\
    0 & 1 & ... & ... \\
    ... & ... & ... & ... \\
    ... & ... &  0  & 1
  \end{array} \left| 
    \begin{array}{ccccc}
      a_{1 ~ m+1} & ... & a_{1q} & ... & a_{1n} \\
      a_{2 ~ m+1} & ... & a_{2q} & ... & a_{2n} \\
      ... & ... & ... & ... & ... \\
      a_{m ~ m+1} & ... & a_{mq} & ... & a_{mn}
    \end{array}
  \right.
\right]=
\left[
  \begin{array}{c}
    x_B \\
    x_N
  \end{array}
\right]=[b]
$$
$$
\begin{array}{c}
  \\ \hline
  x_1 \\
  ... \\
  ... \\
  x_m \\ \hline
  ~
\end{array}
\left|
  \begin{array}{cccc}
    x_1 & x_2 & ... & x_m \\ \hline
    1 & ... & ... & ... \\
    0 & 1 & ... & ... \\
    ... & ... & ... & ... \\
    ... & ... &  0  & 1\\ \hline
        & & & 
  \end{array}
  \begin{array}{ccccc}
    x_{m+1} & ... & x_q & ... & x_n \\ \hline
    a_{1 ~ m+1} & ... & a_{1q} & ... & a_{1n} \\
    a_{2 ~ m+1} & ... & a_{2q} & ... & a_{2n} \\
    ... & ... & ... & ... & ... \\
    a_{m ~ m+1} & ... & a_{mq} & ... & a_{mn} \\ \hline
            & & & &
  \end{array}
\right|
\begin{array}{c}
  b \\ \hline
  b_1 \\
  ... \\
  ... \\
  b_m \\ \hline
  ~
\end{array}
$$
Supongamos que $x_q$ mejora la solución actual, entonces $x_q$ quiere entrar en la base. \\ ¿Qué variable básica $x_1, ..., x_m$ debe salir?
$$b_1(a_1)+b_2(a_2)+...+b_m(a_m)=b $$
$$b_{1~q}(a_1)+b_{2~q}(a_2)+...+b_{m~q}(a_m)=b $$
$$
b_1a_1+b_2a_2+...+b_na_m=b ~ (1)
$$
$$
a_{1~q}a_1+a_{2~q}a_2+...+a_{n~q}a_m=a_q ~(2)
$$
Sea $\varepsilon\geq0$ y realicemos la siguiente operación:
$$(1)-\varepsilon(2)$$
$$(b_1-\varepsilon a_{1 ~ q})a_1+(b_2-\varepsilon a_{2 ~ q})+...+(b_m-\varepsilon a_{m ~q})a_m=b-\varepsilon a_q$$
$$(b_1-\varepsilon a_{1 ~ q})a_1+(b_2-\varepsilon a_{2 ~ q})+...+(b_m-\varepsilon a_{m ~q})a_m+\varepsilon a_q=b$$
Definamos ahora el vector:
$$
x_\varepsilon=\left(
  \begin{array}{c}
    b_1-\varepsilon a_{1 ~ q} \\
    b_2-\varepsilon a_{2 ~ q} \\ 
    ... \\
    b_m-\varepsilon a_{m ~ q} \\
    ... \\
    \varepsilon \\
    ... \\
    0
  \end{array}
\right)
\left(
  \begin{array}{c}
    1 \\
    2 \\
    ... \\
    m \\
    ... \\
    q \\
    ... \\
    0
  \end{array}
\right)
$$
\textbf{Caso I: } Si $a_{iq}<0$ para $1\leq i \leq m$, entonces $x_\varepsilon \in R$ para todo $\varepsilon \geq 0$. \\
En este caso, el problema es no acotado, encuentro una solución que hace que la función objetivo mejore todo lo queramos. \\
\textbf{Caso II: } Si existen algunos $a_{iq}>0$, queremos que:
$$b_i-\varepsilon a_{iq}=0$$
Tomando $$\varepsilon=\min\{\frac{b_i}{a_{iq}} : a_{iq} > 0\}$$ \\
Entonces entra en la base la variable $x_q$, sale de la base la variable asociada a la ecuación $b_i-\varepsilon a_{iq}=0$ (En caso de empate, puede salir la que queramos)
\subsubsection{Forma alternativa de la fase II}
Supongamos que $x_q$ entra en la base y que $x_k$ es la variable básica que debe salir.
$$
\begin{array}{c}
  \\ \hline
  x_1 \\
  ... \\
  ... \\
  \leftarrow x_k \\
  ... \\
  x_m \\ \hline
  ~
\end{array}
\left|
  \begin{array}{cccccc}
    x_1 & x_2 & ... & x_k & ... & x_m \\ \hline
    1 & ... & ... &...& ...& ... \\
    0 & 1 & ... & ...& ...& ... \\
    ... & ... & ... & ... & ...& ... \\
    ... & ... & ... & 1 & ...& ... \\
    ... & ... & ... & ... & ...& ... \\
    ... & ...& .... & ... &  0  & 1\\ \hline
        & & & 
  \end{array}
  \begin{array}{ccccc}
    x_{m+1} & ... & x_q & ... & x_n \\ \hline
    a_{1 ~ m+1} & ... & a_{1q} & ... & a_{1n} \\
    a_{2 ~ m+1} & ... & a_{2q} & ... & a_{2n} \\
    ... & ... & ... & ...& ... \\
    ... & ... & a_{kq} & ...& ... \\
    ... & ... & ... & ...& ... \\
    a_{m ~ m+1} & ... & a_{mq} & ... & a_{mn} \\ \hline
            & & & &
  \end{array}
\right|
\begin{array}{c}
  b \\ \hline
  b_1 \\
  ... \\
  ... \\
  b_k \\
  ... \\
  b_m \\ \hline
  ~
\end{array}
$$
Aplicamos operaciones Gaussianas
$$F_k'=\frac{1}{a_{kq}}F_k$$

$$
\begin{array}{c}
  \\ \hline
  x_1 \\
  ... \\
  ... \\
  \leftarrow x_k \\
  ... \\
  x_m \\ \hline
  ~
\end{array}
\left|
  \begin{array}{cccccc}
    x_1 & x_2 & ... & x_k & ... & x_m \\ \hline
    1 & ... & ... &...& ...& ... \\
    0 & 1 & ... & ...& ...& ... \\
    ... & ... & ... & ... & ...& ... \\
    ... & ... & ... & \frac{1}{a_{kq}}  & ...& ... \\
    ... & ... & ... & ... & ...& ... \\
    ... & ...& .... & ... &  0  & 1\\ \hline
        & & & 
  \end{array}
  \begin{array}{ccccc}
    x_{m+1} & ... & x_q & ... & x_n \\ \hline
    a_{1 ~ m+1} & ... & a_{1q} & ... & a_{1n} \\
    a_{2 ~ m+1} & ... & a_{2q} & ... & a_{2n} \\
    ... & ... & ... & ...& ... \\
    ... & ... & \frac{a_{kq}}{a_{kq}} & ...& ... \\
    ... & ... & ... & ...& ... \\
    a_{m ~ m+1} & ... & a_{mq} & ... & a_{mn} \\ \hline
            & & & &
  \end{array}
\right|
\begin{array}{c}
  b \\ \hline
  b_1 \\
  ... \\
  ... \\
  \frac{b_k}{a_{kq}} \\
  ... \\
  b_m \\ \hline
  ~
\end{array}
$$
Hacemos ahora la siguiente operación:
$$F_j=F_j - a_{jq}F_k ~ \forall j\neq q$$

$$
\begin{array}{c}
  \\ \hline
  x_1 \\
  ... \\
  ... \\
  \leftarrow x_k \\
  ... \\
  x_m \\ \hline
  ~
\end{array}
\left|
  \begin{array}{cccccc}
    x_1 & x_2 & ... & x_k & ... & x_m \\ \hline
    1 & ... & ... &...& ...& ... \\
    0 & 1 & ... & ...& ...& ... \\
    ... & ... & ... & ... & ...& ... \\
    ... & ... & ... & ...  & ...& ... \\
    ... & ... & ... & ... & ...& ... \\
    ... & ...& .... & ... &  0  & 1\\ \hline
        & & & 
  \end{array}
  \begin{array}{ccccc}
    x_{m+1} & ... & x_q & ... & x_n \\ \hline
    a_{1 ~ m+1} & ... & a_{1q} & ... & a_{1n} \\
    a_{2 ~ m+1} & ... & a_{2q} & ... & a_{2n} \\
    ... & ... & ... & ...& ... \\
    ... & ... & 1 & ...& ... \\
    ... & ... & ... & ...& ... \\
    a_{m ~ m+1} & ... & a_{mq} & ... & a_{mn} \\ \hline
            & & & &
  \end{array}
\right|
\begin{array}{c}
  b \\ \hline
  b_1 - a_{1q}\frac{b_k}{a_{kq}}\\
  ... \\
  ... \\
  \frac{b_k}{a_{kq}} \\
  ... \\
  b_m- a_{mq}\frac{b_k}{a_{kq}} \\ \hline
  ~
\end{array}
$$
Como $b_i \geq 0$ entonces, $a_{kq}>0$ de modo que 
$$ b_i-a_{iq}\frac{b_k}{a_{kq}} \geq 0 $$
$$ b_i\geq a_{iq}\frac{b_k}{a_{kq}} $$
\begin{enumerate}
  \item Si $a_{iq}$ para todo $i=1,...,m$ para $i \neq k$ entra la nueva solución factible
  \item Si existen algunos $a_{iq}>0$
  $$ b_i-a_{iq}\frac{b_k}{a_{kq}} \geq 0 $$
  $$ b_i\geq a_{iq}\frac{b_k}{a_{kq}} $$
  Dado que $ a_{iq}>0$
  $$ \frac{b_i}{ a_{iq}}\geq\frac{b_k}{a_{kq}} $$
\end{enumerate}
La fila asociada a $\displaystyle\min\{\frac{b_i}{a_{iq}}: a_{iq}>0\}$ indica la variable que debe salir.


\subsection{¿Qué variable no básica mejora a la solución actual?}
Partimos del siguiente sistema:
$$
\begin{array}{c}
  \\ \hline
  x_1 \\
  ... \\
  ... \\
  x_k \\
  ... \\
  x_m \\ \hline
  ~
\end{array}
\left|
  \begin{array}{cccccc}
    x_1 & x_2 & ... & x_k & ... & x_m \\ \hline
    1 & ... & ... &...& ...& ... \\
    0 & 1 & ... & ...& ...& ... \\
    ... & ... & ... & ... & ...& ... \\
    ... & ... & ... & 1 & ...& ... \\
    ... & ... & ... & ... & ...& ... \\
    ... & ...& .... & ... &  0  & 1\\ \hline
        & & & 
  \end{array}
  \begin{array}{ccccc}
    x_{m+1} & ... & x_q & ... & x_n \\ \hline
    a_{1 ~ m+1} & ... & a_{1q} & ... & a_{1n} \\
    a_{2 ~ m+1} & ... & a_{2q} & ... & a_{2n} \\
    ... & ... & ... & ...& ... \\
    ... & ... & a_{kq} & ...& ... \\
    ... & ... & ... & ...& ... \\
    a_{m ~ m+1} & ... & a_{mq} & ... & a_{mn} \\ \hline
            & & & &
  \end{array}
\right|
\begin{array}{c}
  b \\ \hline
  b_1 \\
  ... \\
  ... \\
  b_k \\
  ... \\
  b_m \\ \hline
  ~
\end{array}
$$
La solución básica asociada a la tabla anterior viene dada por:
$$
x_0=\left(
  \begin{array}{c}
    b_1 \\
    b_2 \\
    ... \\
    b_m \\
    0 \\
    ... \\
    0
  \end{array}
\right)
$$
Sea $f(x_1, x_2, ..., x_n)=c_1 x_1+...+c_n x_n$ para $c_i \in \mathbb{R}$ para todo $i=1, ..., n$ \\
\textbf{¿Qué valor alcanza $x_0$ en la función objetivo?}
$$f(x_1, x_2, ..., x_n)=c_1 x_1+...+c_m x_m+c_{m+1} \cdot 0+...+c_n \cdot 0=z_0$$
Vamos a despejar los $x_i$ para $i=1, ..., m$ en términos del resto de variables.
$$
\left\{
  \begin{array}{c}
    \displaystyle x_1=b_1-\sum_{j=m+1}^{n}a_{1j}x_j \\
    \displaystyle x_2=b_2-\sum_{j=m+1}^{n}a_{2j}x_j \\
    ... \\
    \displaystyle x_m=b_m-\sum_{j=m+1}^{n}a_{mj}x_j \\
  \end{array}
\right.
$$ 
\textbf{¿Qué valor alcanza la función objetivo en la solución general $x$?}
$$f(x_1, x_2, ..., x_n)=c_1 (b_1-\sum_{j=m+1}^{n}a_{1j}x_j)+c_2 (b_2-\sum_{j=m+1}^{n}a_{2j}x_j)+...+c_m (b_m-\sum_{j=m+1}^{n}a_{mj}x_j)+x_{m+1}c_{m+1}+...+c_n x_n$$
Aplicando la propiedad distributiva ahora:
$$f(x_1, x_2, ..., x_n)=c_1 (b_1-\sum_{j=m+1}^{n}a_{1j}x_j)+c_2 (b_2-\sum_{j=m+1}^{n}a_{2j}x_j)+...+c_m (b_m-\sum_{j=m+1}^{n}a_{mj}x_j)+x_{m+1}c_{m+1}+...+c_n x_n=$$
$$=c_1b_1+c_2+...+c_mb_m-c_1\sum_{j=m+1}^{n}a_{1j}x_j-c_2\sum_{j=m+1}^{n}a_{2j}x_j-...-c_m\sum_{j=m+1}^{n}a_{mj}x_j+x_{m+1}c_{m+1}+...+c_n x_n=$$
$$=z_0+\left[c_{m+1}-\sum_{i=1}^{m}c_i a_{i~m+1}\right]x_{m+1}+\left[c_{m+2}-\sum_{i=1}^{m}c_ia_{i~m+2}\right]x_{m+2}+...+\left[c_n-\sum_{i=1}^{m}c_ia_{in}\right]x_n=$$
$$=z_0-\left[\sum_{i=1}^{m}c_i a_{i~m+1}-c_{m+1}\right]x_{m+1}-\left[\sum_{i=1}^{m}c_ia_{i~m+2}-c_{m+2}\right]x_{m+2}-...-\left[\sum_{i=1}^{m}c_ia_{in}-c_n\right]x_n=$$
Definimos $r_q=\displaystyle \sum_{i=1}^{m}c_ia_{i~q}-c_{q}$ como costo relativo asociado a la variable $x_i$. \\
Si el problema es de minimizar, queremos que todos los $r_q<0$.\\
Por otro lado, si el problema es de maximizar queremos que todos los $r_q>0$, de esta manera aumentamos el valor objetivo. \\
Por tanto, salen aquellos que no cumplan la condición de optimalidad.

\subsection{Teorema de condición necesaria y suficiente de
  optimalidad}

La discusión anterior se puede ver de forma resumida en el siguiente
teorema:
\begin{teorema}
  Dado un PPL 
  $$min (max) ~ c^t x $$
  $$s.a \left\{
    \begin{array}{c}
      Ax=b \\
      x\geq 0
    \end{array}
  \right.$$
  Si $x_0$ es una solución básica factible de [P] entonces
  \begin{enumerate}
    \item Si todos los costos relativos $r_j$ de las variables básica son positivas o nulas (negativas o nulas) en el caso de maximizar (en el caso de minimizar) entonces $x_0$ es solución óptima de [P].
    \item Si alguno de los costos relativos es negativo (positivo) en el caso de maximizar (minimizar) entonces la solución actual no es óptima y la variable asociada a dicho coste relativo mejorará la solución actual.
    \begin{enumerate}
      \item Si existe $\varepsilon\{\frac{b_i}{a_{ij}} : a_{ij} >0 \}$ entonces la variable que debe de salir es aquella asociada a la fila donde se alcanza el mínimo, $\varepsilon$
      \item Si todos los componentes $a_{ij}<0$ asociados a la variable que \textit{entrar} entrar entonces el problema es no acotado. Podemos encontrar entonces una dirección de ilimitación.
    \end{enumerate}
  \end{enumerate}
  Recordemos, $$r_j=\sum_{i=1}^{m}c_i a_{ij}-c_j$$ 
  Los costos relativos asociados a las variables básicas siempre es $0$.
\end{teorema}

\begin{ejemplo}
  $$  max ~ 3x_1+x_2+3x_3$$
  $$ s.a \left\{
    \begin{array}{c}
      2x_1+x_2+x_3\leq2\\
      x_1+2x_2+3x_3\leq 5 \\
      2x_1+2x_2+x_3\leq 6
    \end{array}
  \right.
  $$
  $$x_1, x_2, x_3\geq 0$$
  Pasamos a formato estándar
  $$  max ~ 3x_1+x_2+3x_3$$
  $$ s.a \left\{
    \begin{array}{c}
      2x_1+x_2+x_3 + s_1=2\\
      x_1+2x_2+3x_3+ s_2 = 5 \\
      2x_1+2x_2+x_3 + s_3=  6
    \end{array}
  \right.
  $$
  $$x_1, x_2, x_3, s_1, s_2, s_3 \geq 0$$
  Construimos la tabla simplex
  $$
  \begin{array}{c|cccccc|c}
    & 3 & 1 & 3 & 0 & 0 & 0 &  \\
    & x_1 & x_2 & x_3 & s_1 & s_2 & s_3 & b \\ \hline
    0 ~ s_1 & 2 & 1 & 1 & 1 & 0 & 0 & 2 \\
    0 ~ s_2 & 1 & 2 & 3 & 0 & 1 & 0 & 5 \\
    0 ~ s_3 & 2 & 2 & 1 & 0 & 0 & -1 & 6 \\ \hline
    & -3 & -1 & -3 & 0 & 0 & 0 & 0
  \end{array}
  $$
  Vemos que la variable $x_1$ y $x_3$ quieren entrar, elegimos una, en este caso $x_1$, y debe salir $s_1$. 
  $$ F_1'=\frac{F_1}{2} $$
  $$
  \begin{array}{c|cccccc|c}
    & 3 & 1 & 3 & 0 & 0 & 0 &  \\
    & x_1 & x_2 & x_3 & s_1 & s_2 & s_3 & b \\ \hline
    3 ~ x_1 & 1 & 1/2 & 1/2 & 1/2 & 0 & 0 & 1 \\
    0 ~ s_2 & 1 & 2 & 3 & 0 & 1 & 0 & 5 \\
    0 ~ s_3 & 2 & 2 & 1 & 0 & 0 & -1 & 6 \\ \hline
    & -3 & -1 & -3 & 0 & 0 & 0 & 0
  \end{array}
  $$

  $$ F_2'=F_2-F_1 $$
  $$ F_3'=F_3-2F_1 $$
  $$
  \begin{array}{c|cccccc|c}
    & 3 & 1 & 3 & 0 & 0 & 0 &  \\
    & x_1 & x_2 & x_3 & s_1 & s_2 & s_3 & b \\ \hline
    3 ~ x_1 & 1 & 1/2 & 1/2 & 1/2 & 0 & 0 & 1 \\
    0 ~ s_2 & 0 & 3/2 & 5/2 & 1/2 & 1 & 0 & 4 \\
    0 ~ s_3 & 0 & 1 & 0 & -1 & 0 & 1 & 4 \\ \hline
    & 0 & 1/2 & -3/2 & 3/2 & 0 & 0 & 3
  \end{array}
  $$

  Debe entrar en la base la variable $x_3$, haciendo cociente, llegamos que la variable que debe salir es $s_2$, por tanto debemos convertir la columna de $x_3$ en la forma $(0, 1, 0)^t$
  $$F_2'=\frac{2}{5}F_2$$

  $$
  \begin{array}{c|cccccc|c}
    & 3 & 1 & 3 & 0 & 0 & 0 &  \\
    & x_1 & x_2 & x_3 & s_1 & s_2 & s_3 & b \\ \hline
    3 ~ x_1 & 1 & 1/2 & 1/2 & 1/2 & 0 & 0 & 1 \\
    3 ~ x_3 & 0 & 3/5 & 1 & -2/10 & 2/5 & 0 & 8/5 \\
    0 ~ s_3 & 0 & 1 & 0 & -1 & 0 & 1 & 4 \\ \hline
    & 0 & 1/2 & -3/2 & 3/2 & 0 & 0 & 3
  \end{array}
  $$

  $$
  F_1'=F_1-\frac{1}{2}F_2$$$  $

  $$
  \begin{array}{c|cccccc|c}
    & 3 & 1 & 3 & 0 & 0 & 0 &  \\
    & x_1 & x_2 & x_3 & s_1 & s_2 & s_3 & b \\ \hline
    3 ~ x_1 & 1 & 1/5 & 0 & 6/10 & -2/5 & 0 & 1 \\
    3 ~ x_3 & 0 & 3/5 & 1 & -2/10 & 2/5 & 0 & 8/5 \\
    0 ~ s_3 & 0 & 1 & 0 & -1 & 0 & 1 & 4 \\ \hline
    & 0 &  7/5 & 0 & 6/5 & 3/5 & 0 & 27/5
  \end{array}
  $$
  Por tanto, la solución óptima es:
  $$ x_0=\left(
    \begin{array}{c}
      \displaystyle \frac{1}{5} \\
      0 \\
      \displaystyle \frac{8}{5} \\
      0 \\
      0 \\
      4
    \end{array}
  \right)
  $$
\end{ejemplo}
\section{Degeneración del ciclado}
Vamos a ver ahora unos casos donde el método simplex nos lleva a un
ciclo,

$$ max ~2x_1+3x_2-x_3-12x_4$$
$$s.a \left\{
  \begin{array}{c}
    -2x_1-9x_2+x_3+9x_4 \leq 0 \\
    1/3x_1+x_2-1/3x_3-2x_4 \leq 0
  \end{array}
\right.
$$
$$x_1, x_2, x_3, x_4 \geq 0$$
Pasamos el problema a formato estándar, y construimos la tabla SIMPLEX:
$$
\begin{array}{c|cccccc|c}
  & 2 & 3 & -1 & -12 & 0 & 0 & \\
  & x_1 & x_2 & x_3 & x_4 & s_1 & s_2 & b \\ \hline
  0 ~ s_1 & -2 & -9 & 1 & 9 & 1 & 0 & 0 \\
  0 ~ s_2 & 1/3 & 1 & 1/3 & -2 & 0 & 1 & 0 \\ \hline
  & -2 & -3 & 1 & 12 & 0 & 0 & 0
\end{array}
$$

$$
\begin{array}{c|cccccc|c}
  & 2 & 3 & -1 & -12 & 0 & 0 & \\
  & x_1 & x_2 & x_3 & x_4 & s_1 & s_2 & b \\ \hline
  0 ~ s_1 & 1 & 0 & -2 & -9 & 1 & 9 & 0 \\
  3 ~ x_2 & 1/3 & 1 & -1/3 & -2 & 0 & 1 & 0 \\ \hline
  & -1 & 0 & 0 & 6 & 0 & 3 & 0
\end{array}
$$

$$
\begin{array}{c|cccccc|c}
  & 2 & 3 & -1 & -12 & 0 & 0 & \\
  & x_1 & x_2 & x_3 & x_4 & s_1 & s_2 & b \\ \hline
  2 ~ x_1 & 1 & 0 & -2 & -9 & 1 & 9 & 0 \\
  3 ~ x_2 & 0 & 1 & 1/3 & 1 & -1/3 & -2 & 0 \\ \hline
  & 0 & 0 & -2 & -3 & 1 & 12 & 0
\end{array}
$$

$$
\begin{array}{c|cccccc|c}
  & 2 & 3 & -1 & -12 & 0 & 0 & \\
  & x_1 & x_2 & x_3 & x_4 & s_1 & s_2 & b \\ \hline
  2 ~ x_1 & 1 & 0 & -2 & -9 & 1 & 9 & 0 \\
  -12 ~ x_4 & 0 & 1 & 1/3 & 1 & -1/3 & -2 & 0 \\ \hline
  & 0 & 3 & -1 & 0 & 0 & 6 & 0
\end{array}
$$
$$...$$
$$
\begin{array}{c|cccccc|c}
  & 2 & 3 & -1 & -12 & 0 & 0 & \\
  & x_1 & x_2 & x_3 & x_4 & s_1 & s_2 & b \\ \hline
  0 ~ s_1 & -2 & -9 & 1 & 9 & 1 & 0 & 0 \\
  0 ~ s_2 & 1/3 & 1 & 1/3 & -2 & 0 & 1 & 0 \\ \hline
  & -2 & -3 & 1 & 12 & 0 & 0 & 0
\end{array}
$$
\subsection{Método anticiclado}
\begin{enumerate}
  \item Elegimos un orden entre las variables.
  \item En caso de empate en los criterios de entrada o de salida de variables, elegimos la variable más pequeña usando el orden definido en la lista de la fase I.
\end{enumerate}

\section{Métodos para resolver problemas con restricciones}
Hay casos que al escribir nuestro problema de programación lineal no
somos capaces de inicializar el método SIMPLEX, pues no somos capaces
de calcular una solución básica, esto pasa en problemas de
programación lineal donde aparecen restricciones con $=$ o $\geq$.

Veamos un ejemplo:

\subsection{Resolución de PPL con restricciones $(=)$ y $(\geq)$}

Sea [P] el siguiente PPL:
$$ min ~ 3x_1+5x_2 $$
$$ s.a \left\{
  \begin{array}{c}
    x_1 \leq 4 \\
    x_2 \leq 6 \\
    3x_1+2x_2 \geq 18
  \end{array}
\right.
$$
Lo pasamos a formato estándar, y construimos la tabla SIMPLEX.
$$
\begin{array}{c|ccccc|c} 
  & x_1 & x_2 & s_1 & s_3 & b \\ \hline
  s_1 & 1 & 0 & 1 & 0 & 0 & 4 \\
  s_2 & 0 & 1 & 0 & 1 & 0 & 6 \\
  s_3 & -3 & -2 & 0 & 0 & 1 & -18 \\ \hline
  & & & & &
\end{array}
$$
Es básica pero no factible, por tanto no puede inicializar el simplex.

\subsection{Método de las penalizaciones}
Modificamos el problema original añadiendo una variable artificial $(z_i)$ por cada restricción de $(=)$ y de $(\geq)$. \\
Dichas variables artificiales tendrán un costo $M>>0$ (muy grande) y entrarán en la función objetivo sumando (restando) si el problema es de minimizar (maximizar). \\
Al problema con las variables artificiales se le denomina problema ampliado. \\

Volviendo al problema anterior, 
$$ min ~ 3x_1+5x_2+M z_3 $$
$$ s.a \left\{
  \begin{array}{c}
    x_1+s_1=4 \\
    x_2 + s_2 = 6 \\
    3x_1+2x_2-s_3+z_3=18
  \end{array}
\right.
$$
Recordemos lo siguiente:
\begin{itemize}
  \item Las variables de holgura sirven para transformar las restricciones ($\leq$) o ($\geq$) en $(=)$
  \item Las variables artificiales sirven para iniciar el método SIMPLEX con una solución básica factible.
\end{itemize}

Observemos que
\begin{itemize}
  \item El problema ampliado no es equivalente al problema original. 
  \item Solo coincide cuando en la solución del problema ampliado las variables artificiales sean $0$, $z_i$ para todo $i$.
  \item Si $x_0$ es solución óptima del problema ampliado y todos los $z_i=0$, entonces $x_0$ es solución óptima del problema original. 
  \item El problema ampliado siempre tiene solución.
  \item Si $x_0$ es solución óptima del problema ampliado y existe algún $z_i \neq 0$ \\
  Entonces el problema original NO tiene solución.
  \item Si el problema ampliado es no acotado entonces el problema original también es no acotado.
\end{itemize} 
Resolvamos entonces:
$$ min ~ 3x_1+5x_2+M z_3 $$
$$ s.a \left\{
  \begin{array}{c}
    x_1+s_1=4 \\
    x_2 + s_2 = 6 \\
    3x_1+2x_2-s_3+z_3=18
  \end{array}
\right.
$$

$$
\begin{array}{c|cccccc|c}
  & 3 & 5 & 0 & 0 & 0 & M & \\
  & x_1 & x_2 & s_1 & s_2 & s_3 & z_3 & b \\ \hline
  0 ~ s_1 & 1 & 0 & 1 & 0 & 0 & 0 & 4 \\
  0 ~ s_2 & 0 & 1 & 0 & 1 & 0 & 0 & 6 \\
  M ~ z_3 & 3 & 2 & 0 & 0 & -1 & 1 & 18 \\ \hline
  & 3M-3 & 2M-5 & 0 & 0 & -M & 0 & 18M
\end{array}
$$
En este caso entra $x_1$ y sale $s_1$
$$F_3'=F_3-3F_1$$

$$
\begin{array}{c|cccccc|c}
  & 3 & 5 & 0 & 0 & 0 & M & \\
  & x_1 & x_2 & s_1 & s_2 & s_3 & z_3 & b \\ \hline
  3 ~ x_1 & 1 & 0 & 1 & 0 & 0 & 0 & 4 \\
  0 ~ s_2 & 0 & 1 & 0 & 1 & 0 & 0 & 6 \\
  M ~ z_3 & 0 & 2 & -3 & 0 & -1 & 1 & 6 \\ \hline
  & 0 & 2M-5 & -3M+3 & 0 & -M & 0 & 6M+12
\end{array}
$$

Entra $x_2$ y sale $z_3$. \\
El lector debe continuar este ejercicio.


\subsection{Método de las dos fases}
Otro método que se puede utilizar para calcular una solución básica es
el método de las dos fases, para ello:
\begin{enumerate}
  \item Fase 1
  \begin{enumerate}
    \item Añadimos una variable artificial $z_i$ para cada restricción de $(=)$ o $(\geq)$ igual que en el método de las penalizaciones.
    \item Cambiamos la función objetivo por
    $$ min ~ \sum_{i=1}^{k}z_i $$
    \item Si el problema de la fase 1 tiene solución optima con alguna variable artificial no nula, entonces el problema original no tiene solución.
    \item Si tiene solución óptima y todas las variables artificiales son cero, entonces pasamos a la fase 2.
  \end{enumerate}
  \item Fase 2
  \begin{enumerate}
    \item Cambiamos la función objetivo de la fase 1 por la función objetivo original y eliminamos las columnas de las variables artificiales.
  \end{enumerate}
\end{enumerate}

\begin{ejemplo}
  
  $$ min ~ 3x_1+5x_2+M z_3 $$
  $$ s.a
  \left\{
    \begin{array}{c}
      x_1+s_1=4 \\
      x_2 + s_2 = 6 \\
      3x_1+2x_2-s_3+z_3=18
    \end{array}
  \right.
  $$
  \textbf{Fase 1}
  $$ min ~ z_3 $$
  $$ s.a \left\{
    \begin{array}{c}
      x_1+s_1=4 \\
      x_2 + s_2 = 6 \\
      3x_1+2x_2-s_3+z_3=18
    \end{array}
  \right.
  $$

  $$
  \begin{array}{c|cccccc|c}
    & 0 & 0 & 0 & 0 & 0 & 1 & \\
    & x_1 & x_2 & s_1 & s_2 & s_3 & z_3 & b \\ \hline
    0 ~ s_1 & 1 & 0 & 1 & 0 & 0 & 0 & 4 \\
    0 ~ s_2 & 0 & 1 & 0 & 1 & 0 & 0 & 6 \\
    1 ~ z_3 & 3 & 2 & 0 & 0 & -1 & 1 & 18 \\ \hline
    & 3 & 2 & 0 & 0 & -1 & 0 & 18
  \end{array}
  $$

  $$
  \begin{array}{c|cccccc|c}
    & 0 & 0 & 0 & 0 & 0 & 1 & \\
    & x_1 & x_2 & s_1 & s_2 & s_3 & z_3 & b \\ \hline
    0 ~ x_1 & 1 & 0 & 1 & 0 & 0 & 0 & 4 \\
    0 ~ s_2 & 0 & 1 & 0 & 1 & 0 & 0 & 6 \\
    1 ~ z_3 & 0 & 2 & -3 & 0 & -1 & 1 & 6 \\ \hline
    & 0 & 2 & -3 & 0 & -1 & 0 & 6
  \end{array}
  $$

  $$
  \begin{array}{c|cccccc|c}
    & 0 & 0 & 0 & 0 & 0 & 1 & \\
    & x_1 & x_2 & s_1 & s_2 & s_3 & z_3 & b \\ \hline
    0 ~ x_1 & 1 & 0 & 1 & 0 & 0 & 0 & 4 \\
    0 ~ s_2 & 0 & 0 & 3/2 & 1 & 1/2 & -1/2 & 3 \\
    1 ~ x_2 & 0 & 1 & -3/2 & 0 & -1/2 & 1/2 & 3 \\ \hline
    & 0 & 0 & 0 & 0 & 0 & -1 & 0 
  \end{array}
  $$
  \textbf{Fase 2: }
  $$
  \begin{array}{c|ccccc|c}
    & 3 & 5 & 0 & 0 & 0 & \\
    & x_1 & x_2 & s_1 & s_2 & s_3 & b \\ \hline
    3 ~ x_1 & 1 & 0 & 1 & 0 & 0 & 4 \\
    0 ~ s_2 & 0 & 0 & 3/2 & 1 & 1/2 & 3 \\
    5 ~ x_2 & 0 & 1 & -3/2 & 0 & -1/2 & 3 \\ \hline
    & 0 & 0 & -9/2 & 0  & 5/2 & 27 
  \end{array}
  $$
\end{ejemplo}

\section{Coste computacional}

Supongamos que tenemos un problema de programación lineal de maximizar
escrito en formato estándar. Vamos a contar el número de operaciones
máximas para resolver un problema de programación lineal.

\subsection{Operaciones de una iteración}
\subsubsection{Cálculo de los costes relativos}
Tenemos que calcular los costes relativos de las variables no básicas,
por tanto por cada variable no básica, calcular el coste relativo nos
cuesta:

\begin{enumerate}
  \item $m$ multiplicaciones.
  \item $m$ sumas.
  \item 1 resta.
\end{enumerate}

Por tanto hay que hacer $2m+1$ operaciones para calcular cada coste
relativo.

Además, tenemos $n-m$ variables no básicas, esto hace un total de:
$$
(2m+1) \cdot (n-m) = 2mn - 2m^2+n-m
$$

Si $n >> m$, se tiene que calcular el coste relativo tiene un coste
$o(mn)$

\subsubsection{Calculo del mínimo de los costes relativos}
Para cálcular el mínimo de los costes relativos tenemos que iterar
entre todos los costes relativos de las variables no básicas, esto es
que tenemos que hacer $n-m$ comparaciones.

\subsection{Operaciones totales}
De esta forma, dado que el método simplex explora todos los puntos
extremos, y el número máximo de puntos extremos viene dado por
$$
{n\choose m}
$$

Supongamos que $n \geq 2m$, entonces:
$$
{n\choose m} = \frac{n!}{m!(n-m)!} \geq \left(\frac{n}{m}\right)^m \geq 2^m
$$

Por tanto, parece lógico temerse que vamos a tener un número de
iteraciones exponencial.

En la práctica se observa, que el orden del método simplex $o(m^2n)$,
para ver casos extremos donde el método simplex alcanza un orden
exponencial se puede ver el ejemplo clásico de Klee y Minty.